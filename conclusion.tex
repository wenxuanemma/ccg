\section{Conclusion}
\label{sec:conclusion}
%In this paper, 
We present \name, a system that enforces customizable network consistency properties 
with high efficiency.
We highlight the network uncertainty problem and its ramifications, 
and propose a network modeling technique correctly derives consistent outputs even in the presence
of uncertainty.
%We argue it is a crucial task to keep network consistent under uncertainty, 
%and propose \name as a framework to ensure customizable consistency notion.
%Any consistency invariant is specified as a black-box plugged into \name.
The core algorithm of \name leverages the uncertainty-aware network model, 
%and the outputs of the specific black-box, 
and synthesizes a feasible network update plan (ordering and timing of control messages).
In addition to ensuring that there are no violations of consistency requirements,
\name also tries to maximize update parallelism, subject to the constraints imposed by the requirements.
Through emulations and experiments on an SDN testbed, 
we show that \name is capable of achieving a better consistency vs.
efficiency trade-off than existing mechanisms.
\cut{Also, it is known~\cite{Mahajan13} that certain update sequences can be
processed without use of external mechanisms. Our experimental results provide
validation of this observation over additional properties and environments. 
In future work, we plan to develop a theoretical framework to analyze
the consistency property space to explore the efficiency-consistency
relationship.}


%Also, from the experimental results, we observe that for some consistency properties, 
%a feasible update plan is always available without tranlated by a heavy mechanism first.
%This motivates our future work to theoretically expore the consistency property space to find out the 
%efficiency-consistency relationship.


%As future work...

