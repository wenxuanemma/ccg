\section{Implementation}
\label{sec:impl}
We implemented a prototype of \name with 8000+ lines of C++ code. \name is
a shim layer between an SDN controller and network devices, 
intercepting and scheduling network updates issued by the
controller in real time. 
%It verifies the updates and synthesizes a correct
%ordering of updates to preserve consistency during network transition. 

\name maintains several types of state, including network-wide data plane
rules, the uncertainty state of each rule, the set of buffered updates, and bandwidth
information (e.g., for congestion-free invariants). It stores data plane rules
within a multi-layer trie in which each layer's sub-trie represents a packet header field.
We designed a customized trie data structure for handling different types of rule wildcards,
e.g., full wildcard, subnet wildcard, or bitmask wildcard~\cite{openflow-spec},
%differently,
%--- rules agnostic to matching on certain bits or fields, 
and a fast one-pass traversal algorithm to accelerate verification.
\wxzcr{
To handle wildcarding for bitmasks, each node in the trie has 
three child branches, one for each of \{0, 1, don't care\}. 
For subnetting, the wildcard branch has no children, but points 
directly to a next layer sub-trie or a rule set.
Thus, unlike other types of trie, 
the depth of subnet wildcard tries is not fixed as the number of bits in this field, 
but instead equals to the longest prefix among all the rules it stores.
Accordingly, traversal cost is reduced compared with general tries.
For full wildcard fields, values can only be non-wildcarded or full wildcarded.
The specialized trie structure for this type of field is a plain binary tree plus a wildcard table.

When a new update arrives, we need to determine the set of affected ECs,
as well as the rules affecting each EC.
VeriFlow~\cite{VeriFlow} performs a similar task via a two-pass algorithm,
first traversing the trie to compute a set of ECs, and then for each of the
discovered ECs, traversing the trie again to extract related rules.  In \name,
using callback functions and depth first searching, 
%we implemented in place checking, and 
the modeling work is finished with only one
traversal. This algorithm eliminates both the unnecessary extra pass over the
trie and the need to allocate memory for intermediate results.}
%we optimize this process by maintaining some additional accounting information,
%which lets us accomplish our similar objective in a single pass.  
%Our algorithm starts from the top layer subtrie, and combinations
%of its branches that match the first field of the update to be checked are
%selected. The traversal continues on the matched combinations, and would be
%further confined by the following fields of the update.  A matched combination
%of branches in the last level is an EC, and it already points to the rule set
%for that EC.  
%In
%addition, this approach results in a smaller number of ECs---each wildcard
%branch itself forms a matched combination.

In addition to forwarding rules, 
the data structure and algorithm are also capable of handling packet transformation rules, 
such as Network Address Translation (NAT) rules, and rules with VLAN tagging, which are used by CU for versioning, and verified by \name when the CU plug-in is triggered (see \S\ref{sec:eval}).

%\wxznew{
%One special case is dealing with packet transformation rules. 
%When such a rule is encountered during graph traversal phase, a new pass over the trie
%for the transformed EC is triggered. 
%Note that the graph is constructed while traversing rules, so only rules encountered 
%before transformation are kept, and the remaining rule set is discarded.
%More importantly, in this case, multiple trie traversals are possible, but only when 
%necessary, i.e., when a transformation rule is possibly forwarding packets for the original EC.
%} 

To keep track of the uncertainty states of rules, we design a compact state machine, which enables \name to detect rules that cause potential race conditions. If desired, our implementation can be configured to insert barrier messages to serialize those rule updates.

To bound the amount of time that the controller is uncertain about network states, 
we implemented two alternate types of the confirmation mechanisms: (1) an application-level acknowledgment by
modifying the user-space switch program in Mininet, and (2) leveraging the
barrier and barrier reply messages for our physical SDN testbed experiments.

\wxzcr{
%The way that \name stores and retrieves data plane rules is motivated by VeriFlow~\cite{VeriFlow}. 
%The essential idea is to build a multiple-layer trie, with each layer sub-trie 
%representing a packet header field. 
%Each level of a sub-trie corresponds to a bit of that \cut{packet header} field.
%%and can be one of three possible values: $0$, $1$, or wildcard.
%An upper layer sub-trie contains pointers on leaves to sub-tries in the next layer. 
%Dataplane rules are stored at the leaves of bottom sub-tries.
%A path from the root to a leaf of a bottom sub-trie determines a packet set, and one or more such sets can be merged together to form an \emph{equivalence class} (EC) of packets, \matt{i.e.,}
%%More formally, an \emph{equivalence class} (EC) is 
%a set of packets experiencing the same behaviors \wxznew{throughout the network}\cut{at any network device}.
%Upon arrival of an update, its effect on the network state is checked, by limiting searching to ECs whose behavior may be affected by this update, and building a graph model for each affected EC.
%However, given the need to store multiple different representations of the network state, the storage and processing overhead of \name could be larger than just maintaining a single snapshot, \matt{as done in} VeriFlow.
%\wxznew{
%Moreover, there are rules that perform packet transformation,
%such as Network Address Translation, i.e., these rules transform packets from one
%EC to another EC. To accurately model network behaviors, we need to able to handle such rules,
%which are left out in our motivating work, Veriflow.
%To meet this end,
%}
%we design a scalable data structure and an efficient algorithm to operate on it.

%\paragraphb{Customized Trie Data Structure}
%To scale the data structure,
%besides the general type of trie, we design two types of customized trie structure. 
%\matt{
%One complication is dealing with {\em wildcards} -- rules agnostic to matching on certain bits
%or fields. Wildcards complicate the traversal process, as multiple branches may need to be traversed, and the manner
%in which they are traversed can depend on the semantics of the rule (e.g., standard wildcards vs. longest-prefix match).
%To address this, we construct an algorithm that handles general bitmasking, then extend it with optimizations
%to more efficiently handle two common-case wildcard patterns: full wildcards (the entire field is wildcarded) and subnet mask (all bits
%less than a certain significance are wildcarded).  }

%\paragraphb{One-pass Traversal Algorithm}

%These efforts together with a highly optimized implementation 
%allow \name to run almost 100X faster compared to VeriFlow
%%despite having all the challenges we listed.
%with 15X less memory overhead% (540MB vs. 9GB).
%(\S\ref{sec:microbenchmark}). 
%One of our ongoing works is exploring the parallel implementation of the data structure.

%Similar to our prior work~\cite{VeriFlow}, 
\name exposes a set of APIs that can be used to write general queries in C++.
The APIs allow the network operator to get a list of affected equivalence 
classes given an arbitrary forwarding rule, the corresponding 
forwarding graphs, as well as traverse these graphs in a controlled manner 
and check properties of interest. For instance, an operator can ensure
packets from an insecure source encounter a firewall before accessing an internal server.
%To ensure this invariant, \name can be extended using the above APIs to incorporate 
%a custom query algorithm that reports an violation when the packet from that source bypasses the firewall.
}

